%%%% Custom macros
%%%% Macros

%% Greek letters

\newcommand{\al}{\alpha}
% \newcommand{\be}{\beta}
\newcommand{\g}{\gamma}
\newcommand{\de}{\delta}
\newcommand{\e}{\epsilon}
\newcommand{\eps}{\varepsilon}
\newcommand{\ka}{\kappa}
\newcommand{\la}{\lambda}
\newcommand{\sig}{\sigma}
\newcommand{\om}{\omega}
\newcommand{\Om}{\Omega}
\let\oldth\th %% \th is used for "thorn"
\renewcommand{\th}{\theta}

%% Tweak some Greek letters
\newcommand{\bchi}{\mbox{\raisebox{.4ex}{\begin{large}$\chi$\end{large}}}}
\newcommand{\Chi}{\mbox{\Large$\chi$}} % nicer looking Chi
\newcommand{\bzeta}{\boldsymbol{\zeta}} % Riemann zeta function
\newcommand{\bxi}{\boldsymbol{\xi}}

%% Blackboard

\newcommand{\NN}{\mathbb{N}}
\newcommand{\ZZ}{\mathbb{Z}}
\newcommand{\QQ}{\mathbb{Q}}
\newcommand{\RR}{\mathbb{R}}
\newcommand{\CC}{\mathbb{C}}
\newcommand{\TT}{\mathbb{T}}
\newcommand{\DD}{\mathbb{D}}
\newcommand{\HH}{\mathbb{H}}
\newcommand{\UU}{\mathbb{U}}
\newcommand{\1}{\mathbbm{1}}

%% Caligraphic

\newcommand{\cA}{\mathcal{A}}
\newcommand{\cB}{\mathcal{B}}
\newcommand{\cC}{\mathcal{C}}
\newcommand{\cD}{\mathcal{D}}
\newcommand{\cE}{\mathcal{E}}
\newcommand{\cF}{\mathcal{F}}
\newcommand{\cG}{\mathcal{G}}
\newcommand{\cH}{\mathcal{H}}
\newcommand{\cI}{\mathcal{I}}
\newcommand{\cJ}{\mathcal{J}}
\newcommand{\cK}{\mathcal{K}}
\newcommand{\cL}{\mathcal{L}}
\newcommand{\cM}{\mathcal{M}}
\newcommand{\cN}{\mathcal{N}}
\newcommand{\cO}{\mathcal{O}}
\newcommand{\cP}{\mathcal{P}}
\newcommand{\cQ}{\mathcal{Q}}
\newcommand{\cR}{\mathcal{R}}
\newcommand{\cS}{\mathcal{S}}
\newcommand{\cT}{\mathcal{T}}
\newcommand{\cU}{\mathcal{U}}
\newcommand{\cV}{\mathcal{V}}
\newcommand{\cW}{\mathcal{W}}


%% Roman, italic, boldface

\newcommand{\bA}{\mathbf{A}}
\newcommand{\bB}{\mathbf{B}}
\newcommand{\bC}{\mathbf{C}}
\newcommand{\bD}{\mathbf{D}}
\newcommand{\bE}{\mathbf{E}}
\newcommand{\bI}{\mathbf{I}}
\newcommand{\bK}{\mathbf{K}}
\newcommand{\bL}{\mathbf{L}}
\newcommand{\bM}{\mathbf{M}}
\newcommand{\bN}{\mathbf{N}}
\newcommand{\bP}{\mathbf{P}}
\newcommand{\bS}{\mathbf{S}}
\newcommand{\bT}{\mathbf{T}}
\newcommand{\bX}{\mathbf{X}}
\newcommand{\ba}{\mathbf{a}}
\newcommand{\bb}{\mathbf{b}}
\newcommand{\bc}{\mathbf{c}}
\newcommand{\bd}{\mathbf{d}}
\newcommand{\be}{\mathbf{e}}
\newcommand{\bg}{\mathbf{g}}
\newcommand{\bh}{\mathbf{h}}
\newcommand{\br}{\mathbf{r}}
\newcommand{\bx}{\mathbf{x}}
\newcommand{\by}{\mathbf{y}}
\newcommand{\bu}{\mathbf{u}}
\newcommand{\bv}{\mathbf{v}}
\newcommand{\bw}{\mathbf{w}}
\newcommand{\bz}{\mathbf{z}}
\newcommand{\bq}{\mathbf{q}}
\newcommand{\bzero}{\mathbf{0}}


%% Principal value integral
\def\Xint#1{\mathchoice
  {\XXint\displaystyle\textstyle{#1}}%
  {\XXint\textstyle\scriptstyle{#1}}%
  {\XXint\scriptstyle\scriptscriptstyle{#1}}%
  {\XXint\scriptscriptstyle\scriptscriptstyle{#1}}%
  \!\int}
\def\XXint#1#2#3{{\setbox0=\hbox{$#1{#2#3}{\int}$}
    \vcenter{\hbox{$#2#3$}}\kern-.5\wd0}}
\def\ddashint{\Xint=}
\def\pvint{\Xint-}

%% Arc over symbols
% reference: https://tex.stackexchange.com/questions/96680/a-better-notation-to-denote-arcs-for-an-american-high-school-textbook
\makeatletter
\DeclareFontFamily{U}{tipa}{}
\DeclareFontShape{U}{tipa}{m}{n}{<->tipa10}{}
\newcommand{\arc@char}{{\usefont{U}{tipa}{m}{n}\symbol{62}}}%
\newcommand{\arc}[1]{\mathpalette\arc@arc{#1}}
\newcommand{\arc@arc}[2]{%
  \sbox0{$\m@th#1#2$}%
  \vbox{
    \hbox{\resizebox{\wd0}{\height}{\arc@char}}
    \nointerlineskip
    \box0
  }%
}


%% colored boxes
\newsavebox{\astrutbox}
\sbox{\astrutbox}{\rule[-5pt]{0pt}{20pt}}
\newcommand{\astrut}{\usebox{\astrutbox}}
\newcommand{\rls}{\raisebox{2pt}{\tikz{\draw[red,solid,line width=0.9pt](0,0) -- (5mm,0);}}}
\newcommand{\rld}{\raisebox{2pt}{\tikz{\draw[red,dashed,line width=1.0pt](0,0) -- (5mm,0);}}}
\newcommand{\bls}{\raisebox{2pt}{\tikz{\draw[blue,solid,line width=0.9pt](0,0) -- (5mm,0);}}}
\newcommand{\bld}{\raisebox{2pt}{\tikz{\draw[blue,dashed,line width=1.0pt](0,0) -- (5mm,0);}}}
\newcommand{\gls}{\raisebox{2pt}{\tikz{\draw[green,solid,line width=0.9pt](0,0) -- (5mm,0);}}}
\newcommand{\gld}{\raisebox{2pt}{\tikz{\draw[green,dashed,line width=1.0pt](0,0) -- (5mm,0);}}}
\newcommand{\mls}{\raisebox{2pt}{\tikz{\draw[magenta,solid,line width=0.9pt](0,0) -- (5mm,0);}}}
\newcommand{\mld}{\raisebox{2pt}{\tikz{\draw[magenta,dashed,line width=1.0pt](0,0) -- (5mm,0);}}}
\newcommand{\cls}{\raisebox{2pt}{\tikz{\draw[cyan,solid,line width=0.9pt](0,0) -- (5mm,0);}}}
\newcommand{\cld}{\raisebox{2pt}{\tikz{\draw[cyan,dashed,line width=1.0pt](0,0) -- (5mm,0);}}}



%% Delimiters, accents, bars, etc

\newcommand{\abs}[1]{\left|#1\right|}
\newcommand{\ceil}[1]{\left\lceil#1\right\rceil}
\newcommand{\floor}[1]{\left\lfloor#1\right\rfloor}
\newcommand{\conj}[1]{\overline{#1}}
\newcommand{\norm}[1]{\left\|#1\right\|}
\newcommand{\Norm}[2]{\left\|#1\right\|_{#2}}
% Improvement of the above two
% example usage: \norm[2]{f} or \Norm[2]{f}{L^2}
\renewcommand{\norm}[2][0]{%
  \ifcase#1\relax
    \left\Vert #2 \right\Vert\or  % 0
    \lVert #2 \rVert\or           % 1
    \bigl\Vert #2 \bigr\Vert\or   % 2
    \Bigl\Vert #2 \Bigr\Vert\or   % 3
    \biggl\Vert #2 \biggr\Vert\or % 4
    \Biggl\Vert #2 \Biggr\Vert    % 5
  \fi}
\renewcommand{\Norm}[3][0]{%
  \ifcase#1\relax
    \left\Vert #2 \right\Vert_{#3}\or  % 0
    \lVert #2 \rVert_{#3}\or           % 1
    \bigl\Vert #2 \bigr\Vert_{#3}\or   % 2
    \Bigl\Vert #2 \Bigr\Vert_{#3}\or   % 3
    \biggl\Vert #2 \biggr\Vert_{#3}\or % 4
    \Biggl\Vert #2 \Biggr\Vert_{#3}    % 5
  \fi}
\newcommand{\avg}[1]{\langle#1\rangle}
\newcommand{\ds}{\displaystyle}


%% Mathematical operators

\DeclareMathOperator{\re}{Re}
\DeclareMathOperator{\im}{Im}
\DeclareMathOperator{\sgn}{sgn}
\DeclareMathOperator{\erf}{erf}
\DeclareMathOperator{\erfc}{erfc}
\DeclareMathOperator{\ii}{i}
\DeclareMathOperator{\dd}{\,d}
\DeclareMathOperator{\eu}{e}
\DeclareMathOperator{\Sp}{Sp}
\newcommand{\del}{\partial}
\newcommand{\tri}{\triangle}
\newcommand{\grad}{\nabla}
\newcommand{\dvg}{\nabla\cdot}
\newcommand{\curl}{\nabla\times}


%% colored texts

\newcommand{\red}[1]{{\color{red}{#1}}}
\newcommand{\blue}[1]{{\color{blue}{#1}}}
\newcommand{\green}[1]{{\color{green}{#1}}}


%%

\newcommand{\emptyframe}{\begin{frame}\end{frame}}
\newcommand{\question}{\textbf{Question.}}
\newcommand{\sqitem}{\item[$\square$]}


%% 06/12/18 addition

\newcommand{\vs}{\vspace{1em}}
\newcommand{\tp}{^{\rm T}}

%% 06/22/18 addition
% for 3607
\newcommand{\meps}{\fbox{eps}}
\newcommand{\flops}{\textit{flops}}
% \setlength{\fboxsep}{1.5pt}
\newcommand\x{\times}
\newcommand\bigzero{\makebox(0,0){\text{\LARGE 0}}}
\newcommand*{\bord}{\multicolumn{1}{c|}{}}


%% Continued numbering over multiple enumerate environments
% https://tex.stackexchange.com/questions/55000/continuing-enumerate-counters-in-beamer
\newcounter{saveenumi}
\newcommand{\seti}{\setcounter{saveenumi}{\value{enumi}}}
\newcommand{\conti}{\setcounter{enumi}{\value{saveenumi}}}


%% Extension to amsmath matrix environment
% improving matrix constructors
% source: http://texblog.net/latex-archive/maths/amsmath-matrix/
\makeatletter
\renewcommand*\env@matrix[1][*\c@MaxMatrixCols c]{%
  \hskip -\arraycolsep
  \let\@ifnextchar\new@ifnextchar
  \array{#1}}
\makeatother

% In order to make the column lines to look nicer:
\setlength\delimitershortfall{0pt}


%% 11/09/18 addition: vertical spaces
\newcommand{\vsone}{\vspace{\stretch{1}}}
\newcommand{\vstwo}{\vspace{\stretch{2}}}
\newcommand{\vsthree}{\vspace{\stretch{3}}}

%% 02/23/19 addition: wide hat and wide tilde
\newcommand{\wh}[1]{\widehat{#1}}
\newcommand{\wt}[1]{\widetilde{#1}}
